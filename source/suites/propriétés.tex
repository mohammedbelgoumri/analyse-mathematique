\section{Propriétés des suites réelles}
    Une différence importante entre \((\reals, +, \cdot)\) et \((\complexes, +, \cdot)\) est le fait que \(\reals\) est totalement ordonné.
    On peut donc parler des suites réelles qui deviennent arbitrairement grandes ou petites quand \(n\to+\infty\), chose qu'on ne peut pas faire avec des suites complexes.

    \begin{definition}
        \ \\
        Soit \((u_n)_{n\in\naturals}\) une suite réelle divergente. On dit que \(u_n\) \emph{tend vers \(+\infty\)} (resp. \(-\infty\)  ssi 
        \[
            \forall A \in \reals, \quad \exists n_0 \in \naturals, \quad \forall n \in \naturals, \quad n > n_0 \Rightarrow u_n > A\ \mathrm{(resp.}\ u_n < A)     
        \]  
        On note :  \(\lim u_n = +\infty\) (resp. \(-\infty\)).
    \end{definition}

    \begin{definition}[Suite minorée, majorée, bornée]\ \\
        
    \end{definition}