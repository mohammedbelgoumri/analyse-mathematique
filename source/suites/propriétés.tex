\section{Propriétés des suites réelles}
    Une différence importante entre \((\reals, +, \cdot)\) et \((\complexes, +, \cdot)\) est le fait que \(\reals\) est totalement ordonné.
    On peut donc parler des suites réelles qui deviennent arbitrairement grandes ou petites quand \(n\to+\infty\), chose qu'on ne peut pas faire avec des suites complexes.

    \begin{definition}
        \ \\
        Soit \((u_n)_{n\in\naturals}\) une suite réelle divergente. On dit que \(u_n\) \emph{tend vers \(+\infty\)} (resp. \(-\infty\)  ssi 
        \[
            \forall A \in \reals, \quad \exists n_0 \in \naturals, \quad \forall n \in \naturals, \quad n > n_0 \Rightarrow u_n > A\ \mathrm{(resp.}\ u_n < A)     
        \]  
        On note :  \(\lim u_n = +\infty\) (resp. \(-\infty\)).
    \end{definition}

    \begin{definition}\ \\
        Soit \((u_n)_{n\in\naturals}\) une suite réelle.
        On dit que \(u_n\) est :
        \begin{itemize}
            \item\emph{Majorée} ssi : \(\exists M \in \reals, \quad \forall n \in \naturals, \quad u_n \le M\).
            \item\emph{Minorée} ssi \(\exists m \in \reals, \quad \forall n \in \naturals, \quad u_n \ge m\).
            \item\emph{Croissante} (resp. \emph{strictement croissante}) ssi : \(\forall n\in \naturals, \quad u_{n+1} \ge u_n\). (resp. \(u_{n+1} > u_n\)).
            \item\emph{Décroissante} (resp. \emph{strictement décroissante}) ssi : \(\forall n\in \naturals, \quad u_{n+1} \le u_n\). (resp. \(u_{n+1} < u_n\)).
            \item\emph{Monotone}(resp. \emph{strictement monotone}) ssi elle est croissante (resp. strictement croissante) ou décroissante (resp. strictement décroissante).
        \end{itemize}
    \end{definition}


    \begin{theorem}[Suites monotones]\ \\
        \label{thm:suites_monotone}
        Soit \(u_n)_{n\in\naturals}\) une suite réelle. 
        \begin{enumerate}[label=(\roman*)]
            \item Si \(u_n\) est croissante et majorée, alors \(\lim u_n = \sup\limits_{n\in\naturals}u_n\).
            \item Si \(u_n\) est croissante et minorée, alors \(\lim u_n = \inf\limits_{n\in\naturals}u_n\).
        \end{enumerate}
    \end{theorem}

    \begin{proof}
        \ \\
        \begin{enumerate}[label=(\roman*)]
            \item Posons \(\lambda\colonequals\sup\limits_{n\in\naturals}u_n\). On a : 
            \[
            \forall \varepsilon > 0, \quad \exists n_0 \in \naturals, \quad \lambda - \varepsilon < u_{n_0} \le \lambda    
            \]
            
            \(u_n\) étant croissante, \(\lambda - \varepsilon < u_{n_0} \Rightarrow \lambda - \varepsilon < u_{n}\) pour tout \(n\ge n_0\).
            Il en suite que :
            \[
            \forall \varepsilon > 0, \quad \exists n_0 \in \naturals, \quad \forall n\in \naturals, \quad \lambda - \varepsilon < u_{n} \le \lambda < \lambda + \varepsilon
            \]
            D'où la conclusion : \(lim u_n = \lambda\).

            \item La démonstration s'obtient en appliquant le raisonnement précédent à \((-u_n)_{n\in\naturals}\).
        \end{enumerate}
    \end{proof}

    \begin{thedef}[Suites adjacentes]\ \\
        \label{thm:suites_adjacentes}
        Soit \((u_n)\) et \(v_n\) deux suites réelles telles que :
        \begin{enumerate}[label=\(\alph*\).]
            \item \((u_n)\) est croissante.
            \item \((v_n)\) est décroissante.
            \item \((u_n - v_n) \to 0\)
        \end{enumerate}
        \((u_n)\) et \(v_n\) sont dites deux suites \emph{adjacentes}. Elles convergent et vérifient \(\lim u_n =  \lim v_n\).
    \end{thedef}