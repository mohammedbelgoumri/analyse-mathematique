\section{Généralités}
    Dans la suite de ce chapitre, \((\mathbb{K}, +, \cdot)\) est l'un des deux corps \((\mathbb{R}, +, \cdot)\) ou \((\mathbb{C}, +, \cdot)\).
    \begin{definition}[Suite numérique]\ \\
        On appelle une \emph{suite numérique} toute application \(u:\mathbb{N} \rightarrow \mathbb{K}\).
        Une suite numérique est notée \((u_n)_{n \in \mathbb{N}}\) plutôt que :
        \[
            \left\{
                \begin{array}{l}
                    u: \mathbb{N} \rightarrow \mathbb{K}\\
                    n \mapsto u_n
                \end{array}
            \right.    
        \]
        \(u_n\) (l'image de \(n\) par cette application) est appelée le \emph{terme général} de la suite.
    \end{definition}

    \begin{definition}[Suite extraite]\ \\
        Soit \((u_n)_{n\in\mathbb{N}}\) une suite numérique. On appelle une suite extraite de \((u_n)_{n\in\mathbb{N}}\) toute suite numérique \((v_n)_{n\in\mathbb{N}}\) de terme général \(v_n = u_{\varphi(n)}\) où \(\varphi: \mathbb{N} \rightarrow \mathbb{N}\) est une application croissante.
    \end{definition}
    
    \begin{thedef}[Limite d'un suite, convergence, divergence]\ \\
        \begin{itemize}
            \item Soit \((u_n)_{n\in\mathbb{N}}\) une suite numérique. Il existe au plus un seul \(l\in\mathbb{K}\) qui vérifie la condition :
            \[
                \forall \varepsilon > 0, \quad \exists n_0 \in \mathbb{N}, \quad \forall n \in \mathbb{N}, \quad n > n_0 \Rightarrow |u_n - l| < \varepsilon    
            \]
            Un tel \(l\) (s'il existe) est appelé la \emph{limite} de \((u_n)_{n\in\mathbb{N}}\). Il est noté \(\lim_{n\to+\infty}u_n\), ou encore \(\lim u_n\).
            \item Une suite est dite \emph{convergente} ssi elle possède une limite \(l \in\mathbb{K}\). Dans ce cas on dit que la suite \emph{converge vers} \(l\).
            \item Une suite qui n'est pas convergente est dite \emph{divergente}.
        \end{itemize}
    \end{thedef}

    \begin{proof}\ \\
        Soit \((u_n)_{n\in\mathbb{N}}\) une suite numérique, \(\varepsilon \in ]0, +\infty[ \) et \(l, l^{\prime} \in \mathbb{K}\) vérifient tous les deux :
        \[
            \begin{array}{l}
                \forall \varepsilon > 0, \quad \exists n_0 \in \mathbb{N}, \quad \forall n \in \mathbb{N}, \quad n > n_0 \Rightarrow |u_n - l| < \varepsilon\\
                \forall \varepsilon > 0, \quad \exists n_0 \in \mathbb{N}, \quad \forall n \in \mathbb{N}, \quad n > n_0 \Rightarrow |u_n - l^{\prime}| < \varepsilon\\
            \end{array}
        \]
        On en déduit l'existence de \(n_1, n_2 \in \mathbb{N}\) tels que :
        \[
            \begin{array}{l}
                \forall n \in \mathbb{N},\quad n > n_1 \Rightarrow |u_n - l| < \frac{\varepsilon}{2}\\
                \forall n \in \mathbb{N},\quad n > n_2 \Rightarrow |u_n - l^{\prime}| < \frac{\varepsilon}{2}\\
            \end{array}    
        \]
        En posant \(n_0 = \max{\{n_1, n_2\}}\), on trouve que :
        \[
            \forall n \in \mathbb{N},\quad n > n_0 \Rightarrow |l - l^{\prime}| = |l - u_n + u_n - l^{\prime}| \le |u_n - l| + |u_n - l^{\prime}| < \varepsilon
        \]
        Autrement dit, on a :
        \[
            \forall \varepsilon > 0, \quad |l - l^{\prime}| < \varepsilon 
        \]
        D'où la conclusion : \(l = l^{\prime}\)
    \end{proof}