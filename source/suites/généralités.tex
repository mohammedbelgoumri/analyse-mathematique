\section{Généralités}
    \begin{definition}[Suite numérique]
        \ \\
        On appelle une \emph{suite numérique} toute application \(u:\naturals \rightarrow \field\).
        Une suite numérique est notée \((u_n)_{n \in \naturals}\) plutôt que :
        \[
            \left\{
                \begin{array}{l}
                    u: \naturals \rightarrow \field\\
                    n \mapsto u_n
                \end{array}
            \right.    
        \]
        \(u_n\) (l'image de \(n\) par cette application) est appelée le \emph{terme général} de la suite.
    \end{definition}

    \begin{definition}[Suite extraite]
        \ \\
        Soit \((u_n)_{n\in\naturals}\) une suite numérique. On appelle une \emph{suite extraite} ou \emph{sous suite} de \((u_n)_{n\in\naturals}\) toute suite numérique \((v_n)_{n\in\naturals}\) de terme général \(v_n = u_{\varphi(n)}\) où \(\varphi: \naturals \rightarrow \naturals\) est une application croissante.
    \end{definition}
    
    \begin{thedef}[Limite d'une suite, convergence, divergence]
        \ \\
        \label{definition:limite_suite}
        \begin{itemize}
            \item Soit \((u_n)_{n\in\naturals}\) une suite numérique. Il existe au plus un seul \(\ell\in\field\) qui vérifie la condition :
            \[
                \forall \varepsilon > 0, \quad \exists n_0 \in \naturals, \quad \forall n \in \naturals, \quad n > n_0 \Rightarrow |u_n - \ell| < \varepsilon    
            \]
            Un tel \(\ell\) (s'il existe) est appelé la \emph{limite} de \((u_n)_{n\in\naturals}\). On le note :  \(\lim\limits_{n\to+\infty}u_n\), ou encore \(\lim u_n\).
            \item Une suite est dite \emph{convergente} ssi elle possède une limite \(\ell \in \field\). Dans ce cas on dit que la suite \emph{converge vers} \(\ell\) et on écrit : \(u_n \to\ell \)
            \item Une suite qui n'est pas convergente est dite \emph{divergente}.
        \end{itemize}
    \end{thedef}

    \begin{proof}
        \ \\
        Soit \((u_n)_{n\in\naturals}\) une suite numérique, \(\varepsilon \in ]0, +\infty[ \) et \(\ell, \ell^{\prime} \in \field\) vérifient tous les deux :
        \[
            \begin{array}{l}
                \forall \varepsilon > 0, \quad \exists n_0 \in \naturals, \quad \forall n \in \naturals, \quad n > n_0 \Rightarrow |u_n - \ell| < \varepsilon\\
                \forall \varepsilon > 0, \quad \exists n_0 \in \naturals, \quad \forall n \in \naturals, \quad n > n_0 \Rightarrow |u_n - \ell^{\prime}| < \varepsilon\\
            \end{array}
        \]
        On en déduit l'existence de \(n_1, n_2 \in \naturals\) tels que :
        \[
            \begin{array}{l}
                \forall n \in \naturals,\quad n > n_1 \Rightarrow |u_n - \ell| < \frac{\varepsilon}{2}\\
                \forall n \in \naturals,\quad n > n_2 \Rightarrow |u_n - \ell^{\prime}| < \frac{\varepsilon}{2}\\
            \end{array}    
        \]
        En posant \(n_0 = \max{\{n_1, n_2\}}\), on trouve que :
        \[
            \forall n \in \naturals,\quad n > n_0 \Rightarrow |\ell - \ell^{\prime}| = |\ell - u_n + u_n - \ell^{\prime}| \le |u_n - \ell| + |u_n - \ell^{\prime}| < \varepsilon
        \]
        Autrement dit, on a :
        \[
            \forall \varepsilon > 0, \quad |\ell - \ell^{\prime}| < \varepsilon 
        \]
        D'où la conclusion : \(\ell =  \ell^{\prime}\)
    \end{proof}

    \begin{theorem}
        \ \\
        Toute suite convergente est bornée. i.e : Si \((u_n)\) est convergente, alors il existe \(M \in \reals^{*}_{+}\) tel que :
        \[\forall n \in \naturals, \quad |u_n| \le M\]
    \end{theorem}

    \begin{proof}
        \ \\
        Soit \((u_n)\) une suite convergente vers \(\ell \in  \field\) et \(\varepsilon > 0\). D'après la définition de la limite, on a :
        \[
            \exists n_0 \in \naturals, \quad \forall n \in \naturals, \quad n > n_0 \Rightarrow |u_n - \ell| < \varepsilon    
        \]
        On en déduit que pour tout \(n>n_0\), on a :
        \[
            |u_n| - |\ell| \le |u_n - \ell| < \varepsilon \Rightarrow |u_n| < |\ell| +\varepsilon     
        \]
        Finalement, en posant \(m = \max{\{|u_n|\ | n \le n_0\}}\) et \(M = \max{\{m, |\ell| + \varepsilon\}}\) on a bien :
        \[\forall n \in \naturals, \quad |u_n| \le M\]
    \end{proof}

    \begin{theorem}[Operations sur les suites convergentes]
        \ \\
        \label{theorem:ops_suites_convergentes}
        Soient \((u_n)\) et \((v_n)\) deux suites convergentes vers \(\ell\in\field\) et \(\ell^{\prime}\in\field\) respectivement. On a :
        \begin{enumerate}[label=(\roman*)]
            \item \((u_n + \lambda v_n)_{n\in\naturals}\) converge vers \(\ell + \lambda \ell^{\prime}\) quelque soit \(\lambda\in\field\). 
            \item \((|u_n|)_{n\in\naturals}\) converge vers \(|\ell|\).
            \item \((\forall n \in \naturals, \quad u_n \le v_n) \Rightarrow \ell \le \ell^{\prime}\)
            \item \((u_nv_n)_{n\in\naturals}\) converge vers \(\ell\ell^{\prime}\).
            \item Si \(\forall n \in \naturals, \quad u_n \neq 0\) et \(\ell\neq 0\), alors \({\left({1\over u_n}\right)}_{n\in\naturals}\) converge vers \({1\over l}\).
            \item Toute suite \((u_{\varphi(n)})_{n\in\naturals}\) extraite de \((u_n)\) converge vers \(\ell\). 
        \end{enumerate}
    \end{theorem}

    \begin{proof}
        % Compléter (iii) jusqu'à (v)
        \ \\
        \begin{enumerate}[label=(\roman*)]
            \item Soit \(\varepsilon>0\). Comme \(u_n\to\ell \) et \(v_n \to \ell^{\prime}\), il existe \(n_1, n_2 \in \naturals\) tels que :
            \[
                \begin{array}{l}
                    \forall n \in \naturals,\quad n > n_1 \Rightarrow |u_n - \ell| < \frac{\varepsilon}{2}\\
                    \forall n \in \naturals,\quad n > n_2 \Rightarrow |v_n - \ell^{\prime}| < \frac{\varepsilon}{2|\lambda|}\\
                \end{array}    
            \]
            En posant \(n_0 = \max{\{n_1, n_2\}}\), on a pour tout \(n>n_0\) :
            \[
                \begin{array}{lcl}
                    |u_n + \lambda v_n - \ell - \lambda \ell^{\prime}| &\le& |u_n  - \ell| + |\lambda|\cdot|v_n - \ell^{\prime}|\\
                     &<& \frac{\varepsilon}{2} + |\lambda|\frac{\varepsilon}{2|\lambda|}\\
                     &=& \varepsilon
                \end{array}
            \]
            Ce qui entraîne que \(\lim (u_n + \lambda v_n) = \ell + \lambda \ell^{\prime}\) 
            
            \item La conclusion suit directement de l'inégalité triangulaire : \[\Big| |u_n| - |\ell| \Big| \le |u_n - \ell|\]
            \item \ 
            \item \ 
            \item \ 
            \item Soit \(\varepsilon > 0\). D'après la définition de la limite, on a un  \(n_0\in\naturals\) tel que : 
            \begin{equation*}
                \tag*{\((\ast)\)}
                \forall n \in \naturals, \quad n > n_0 \Rightarrow |u_n - \ell| < \varepsilon 
            \end{equation*}
            Or, \(\varphi\) est croissante. Elle vérifie donc \(\varphi(n) \ge n\) pour tout \(n\in\naturals\). Il en résulte en utilisant \((\ast)\) que :
            \[
                \forall n \in \naturals, \quad n > n_0 \Rightarrow |u_{\varphi(n)} - \ell| < \varepsilon     
            \]
            Autrement dit : \(u_{\varphi(n)} \to\ell \)
        \end{enumerate}
    \end{proof}

    \begin{corollary}\ \\
        Une suite complexe converge ssi sa partie réelle et sa partie imaginaire convergent. Dans ce cas, on a :
        \[\lim u_n = \lim \Re{(u_n)} + i\ \lim\Im{(u_n)}\] 
    \end{corollary}