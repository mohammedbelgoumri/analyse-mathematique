\section{Généralités}
    Dans la suite de ce chapitre, \((\mathbb{K}, +, \cdot)\) est l'un des deux corps \((\mathbb{R}, +, \cdot)\) ou \((\mathbb{C}, +, \cdot)\).
    \begin{definition}[Suite numérique]\ \\
        On appelle une \emph{suite numérique} toute application \(u:\mathbb{N} \rightarrow \mathbb{K}\).
        Une suite numérique est notée \((u_n)_{n \in \mathbb{N}}\) plutôt que:
        \[
            \left\{
                \begin{array}{l}
                    u: \mathbb{N} \rightarrow \mathbb{K}\\
                    n \mapsto u_n
                \end{array}
            \right.    
        \]
        \(u_n\) (l'image de \(n\) par cette application) est appellé le \emph{terme géneral} de la suite.
    \end{definition}