\section{Généralités}
    Dans la suite de ce chapitre, \((\mathbb{K}, +, \cdot)\) est l'un des deux corps \((\mathbb{R}, +, \cdot)\) ou \((\mathbb{C}, +, \cdot)\).
    
    \begin{definition}[Suite numérique]
        \ \\
        On appelle une \emph{suite numérique} toute application \(u:\mathbb{N} \rightarrow \mathbb{K}\).
        Une suite numérique est notée \((u_n)_{n \in \mathbb{N}}\) plutôt que :
        \[
            \left\{
                \begin{array}{l}
                    u: \mathbb{N} \rightarrow \mathbb{K}\\
                    n \mapsto u_n
                \end{array}
            \right.    
        \]
        \(u_n\) (l'image de \(n\) par cette application) est appelée le \emph{terme général} de la suite.
    \end{definition}

    \begin{definition}[Suite extraite]
        \ \\
        Soit \((u_n)_{n\in\mathbb{N}}\) une suite numérique. On appelle une \emph{suite extraite} ou \emph{sous suite} de \((u_n)_{n\in\mathbb{N}}\) toute suite numérique \((v_n)_{n\in\mathbb{N}}\) de terme général \(v_n = u_{\varphi(n)}\) où \(\varphi: \mathbb{N} \rightarrow \mathbb{N}\) est une application croissante.
    \end{definition}
    
    \begin{thedef}[Limite d'une suite, convergence, divergence]
        \ \\
        \label{definition:limite_suite}
        \begin{itemize}
            \item Soit \((u_n)_{n\in\mathbb{N}}\) une suite numérique. Il existe au plus un seul \(l\in\mathbb{K}\) qui vérifie la condition :
            \[
                \forall \varepsilon > 0, \quad \exists n_0 \in \mathbb{N}, \quad \forall n \in \mathbb{N}, \quad n > n_0 \Rightarrow |u_n - l| < \varepsilon    
            \]
            Un tel \(l\) (s'il existe) est appelé la \emph{limite} de \((u_n)_{n\in\mathbb{N}}\). On le note :  \(\lim\limits_{n\to+\infty}u_n\), ou encore \(\lim u_n\).
            \item Une suite est dite \emph{convergente} ssi elle possède une limite \(l \in\mathbb{K}\). Dans ce cas on dit que la suite \emph{converge vers} \(l\) et on écrit : \(u_n \to l\)
            \item Une suite qui n'est pas convergente est dite \emph{divergente}.
        \end{itemize}
    \end{thedef}

    \begin{proof}
        \ \\
        Soit \((u_n)_{n\in\mathbb{N}}\) une suite numérique, \(\varepsilon \in ]0, +\infty[ \) et \(l, l^{\prime} \in \mathbb{K}\) vérifient tous les deux :
        \[
            \begin{array}{l}
                \forall \varepsilon > 0, \quad \exists n_0 \in \mathbb{N}, \quad \forall n \in \mathbb{N}, \quad n > n_0 \Rightarrow |u_n - l| < \varepsilon\\
                \forall \varepsilon > 0, \quad \exists n_0 \in \mathbb{N}, \quad \forall n \in \mathbb{N}, \quad n > n_0 \Rightarrow |u_n - l^{\prime}| < \varepsilon\\
            \end{array}
        \]
        On en déduit l'existence de \(n_1, n_2 \in \mathbb{N}\) tels que :
        \[
            \begin{array}{l}
                \forall n \in \mathbb{N},\quad n > n_1 \Rightarrow |u_n - l| < \frac{\varepsilon}{2}\\
                \forall n \in \mathbb{N},\quad n > n_2 \Rightarrow |u_n - l^{\prime}| < \frac{\varepsilon}{2}\\
            \end{array}    
        \]
        En posant \(n_0 = \max{\{n_1, n_2\}}\), on trouve que :
        \[
            \forall n \in \mathbb{N},\quad n > n_0 \Rightarrow |l - l^{\prime}| = |l - u_n + u_n - l^{\prime}| \le |u_n - l| + |u_n - l^{\prime}| < \varepsilon
        \]
        Autrement dit, on a :
        \[
            \forall \varepsilon > 0, \quad |l - l^{\prime}| < \varepsilon 
        \]
        D'où la conclusion : \(l = l^{\prime}\)
    \end{proof}

    \begin{theorem}
        \ \\
        Toute suite convergente est bornée. i.e : Si \((u_n)\) est convergente, alors il existe \(M \in \mathbb{R}^{*}_{+}\) tel que :
        \[\forall n \in \mathbb{N}, \quad |u_n| \le M\]
    \end{theorem}

    \begin{proof}
        \ \\
        Soit \((u_n)\) une suite convergente vers \(l \in \mathbb{K}\) et \(\varepsilon > 0\). D'après la définition de la limite, on a :
        \[
            \exists n_0 \in \mathbb{N}, \quad \forall n \in \mathbb{N}, \quad n > n_0 \Rightarrow |u_n - l| < \varepsilon    
        \]
        On en déduit que pour tout \(n>n_0\), on a :
        \[
            |u_n - l| \le |u_n| - |l| < \varepsilon \Rightarrow |u_n| < |l| +\varepsilon     
        \]
        Finalement, en posant \(m = \max{\{|u_n|\ | n \le n_0\}}\) et \(M = \max{\{m, |l| + \varepsilon\}}\) on a bien :
        \[\forall n \in \mathbb{N}, \quad |u_n| \le M\]
    \end{proof}

    \begin{theorem}[Operations sur les suites convergentes]
        \ \\
        \label{theorem:ops_suites_convergentes}
        Soient \((u_n)\) et \((v_n)\) deux suites convergentes vers \(l\in\mathbb{K}\) et \(l^{\prime}\in\mathbb{K}\) respectivement. On a :
        \begin{enumerate}[label=(\roman*)]
            \item \((u_n + \lambda v_n)_{n\in\mathbb{N}}\) converge vers \(l + \lambda l^{\prime}\) quelque soit \(\lambda\in\mathbb{K}\). 
            \item \((|u_n|)_{n\in\mathbb{N}}\) converge vers \(|l|\).
            \item \((\forall n \in \mathbb{N}, \quad u_n \le v_n) \Rightarrow l \le l^{\prime}\)
            \item \((u_nv_n)_{n\in\mathbb{N}}\) converge vers \(ll^{\prime}\).
            \item Si \(\forall n \in \mathbb{N}, \quad u_n \neq 0\) et \(l\neq 0\), alors \({\left({1\over u_n}\right)}_{n\in\mathbb{N}}\) converge vers \({1\over l}\).
            \item Toute suite \((u_{\varphi(n)})_{n\in\mathbb{N}}\) extraite de \((u_n)\) converge vers \(l\). 
        \end{enumerate}
    \end{theorem}

    \begin{proof}
        % Compléter (iii) jusqu'à (v)
        \ \\
        \begin{enumerate}[label=(\roman*)]
            \item Soit \(\varepsilon>0\). Comme \(u_n\to l\) et \(v_n \to l^{\prime}\), il existe \(n_1, n_2 \in \mathbb{N}\) tels que :
            \[
                \begin{array}{l}
                    \forall n \in \mathbb{N},\quad n > n_1 \Rightarrow |u_n - l| < \frac{\varepsilon}{2}\\
                    \forall n \in \mathbb{N},\quad n > n_2 \Rightarrow |v_n - l^{\prime}| < \frac{\varepsilon}{2|\lambda|}\\
                \end{array}    
            \]
            En posant \(n_0 = \max{\{n_1, n_2\}}\), on a pour tout \(n>n_0\) :
            \[
                \begin{array}{lcl}
                    |u_n + \lambda v_n - l - \lambda l^{\prime}| &\le& |u_n  - l| + |\lambda|\cdot|v_n - l^{\prime}|\\
                     &<& \frac{\varepsilon}{2} + |\lambda|\frac{\varepsilon}{2|\lambda|}\\
                     &=& \varepsilon
                \end{array}
            \]
            Ce qui entraîne que \(\lim (u_n + \lambda v_n) = l + \lambda l^{\prime}\) 
            
            \item La conclusion suit directement de l'inégalité triangulaire : \[\Big| |u_n| - |l| \Big| \le |u_n - l|\]
            \item \ 
            \item \ 
            \item \ 
            \item Soit \(\varepsilon > 0\). D'après la définition de la limite, on a un  \(n_0\in\mathbb{N}\) tel que : 
            \begin{equation*}
                \tag*{(*)}
                \forall n \in \mathbb{N}, \quad n > n_0 \Rightarrow |u_n - l| < \varepsilon 
            \end{equation*}
            Or, \(\varphi\) est croissante. Elle vérifie donc \(\varphi(n) \ge n\) pour tout \(n\in\mathbb{N}\). Il en résulte en utilisant (*) que :
            \[
                \forall n \in \mathbb{N}, \quad n > n_0 \Rightarrow |u_{\varphi(n)} - l| < \varepsilon     
            \]
            Autrement dit : \(u_{\varphi(n)} \to l\)
        \end{enumerate}
    \end{proof}

    \begin{corollary}
        Une suite complexe converge ssi sa partie réelle et sa partie imaginaire convergent. Dans ce cas, on a :
        \[\lim u_n = \lim \Re{u_n} + i\ \lim\Im{u_n}\] 
    \end{corollary}